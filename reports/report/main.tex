% This must be in the first 5 lines to tell arXiv to use pdfLaTeX, which is strongly recommended.
\pdfoutput=1
% In particular, the hyperref package requires pdfLaTeX in order to break URLs across lines.

\documentclass[11pt]{article}

\usepackage[usenames,dvipsnames]{xcolor}
% Remove the "review" option to generate the final version.
% \usepackage[review]{acl}
\usepackage[]{acl}

% Standard package includes
\usepackage{times}
\usepackage{latexsym}

% For proper rendering and hyphenation of words containing Latin characters (including in bib files)
\usepackage[T1]{fontenc}
% For Vietnamese characters
% \usepackage[T5]{fontenc}
% See https://www.latex-project.org/help/documentation/encguide.pdf for other character sets

% This assumes your files are encoded as UTF8
\usepackage[utf8]{inputenc}

% This is not strictly necessary, and may be commented out,
% but it will improve the layout of the manuscript,
% and will typically save some space.
\usepackage{microtype}

% If the title and author information does not fit in the area allocated, uncomment the following
%
%\setlength\titlebox{<dim>}
%
% and set <dim> to something 5cm or larger.

\title{CLAfICLe: Cross Lingual Adaptation for In-Context Learning}

% Author information can be set in various styles:
% For several authors from the same institution:
% \author{Author 1 \and ... \and Author n \\
%         Address line \\ ... \\ Address line}
% if the names do not fit well on one line use
%         Author 1 \\ {\bf Author 2} \\ ... \\ {\bf Author n} \\
% For authors from different institutions:
% \author{Author 1 \\ Address line \\  ... \\ Address line
%         \And  ... \And
%         Author n \\ Address line \\ ... \\ Address line}
% To start a seperate ``row'' of authors use \AND, as in
% \author{Author 1 \\ Address line \\  ... \\ Address line
%         \AND
%         Author 2 \\ Address line \\ ... \\ Address line \And
%         Author 3 \\ Address line \\ ... \\ Address line}

\author{Giulio Starace \\
  University of Amsterdam / Amsterdam, The Netherlands \\
  \texttt{giulio.starace@gmail.com} \\}

% user settings
\usepackage{graphicx} % for images
\usepackage{subcaption} % for subfigures
\graphicspath{{../figures/}}
\usepackage[export]{adjustbox}
\usepackage{geometry}
\usepackage{booktabs}


\begin{document}
\maketitle
\begin{abstract}
	This document is a supplement to the general instructions for *ACL authors. It contains instructions for using the \LaTeX{} style files for ACL conferences.
	The document itself conforms to its own specifications, and is therefore an example of what your manuscript should look like.
	These instructions should be used both for papers submitted for review and for final versions of accepted papers.
\end{abstract}

\section{Introduction}

\section{Related Work}

\section{Method}\label{sec:method}
\begin{figure*}[ht]
	\subcaptionbox{\label{fig:sandwich}}
	{\includegraphics{sandwich.pdf}}
	\hfill
	\subcaptionbox{\label{fig:metaicl-gewechselt}}
	{\includegraphics{metaicl-gewechselt.pdf}}
	\hfill
	\subcaptionbox{}
	{\includegraphics{gpt2-gewechselt+metaicla.pdf}}
	\hfill
	\subcaptionbox{}
	{\includegraphics{gpt2-gewechselt+metaiclva.pdf}}
	\caption{Overview of each of the models evaluated in one of the two TGT
		languages (French or German). The baseline
		\textcolor[HTML]{79d6ae}{Sandwich} model (\subref{fig:sandwich}) sandwiches
		\textcolor[HTML]{332345}{MetaICL} \citep{min_metaicl_2022} (which we
		separately evaluate only in English) between two complementary translation
		API calls. \textcolor[HTML]{38AAAC}{MetaICL-geWECHSELt}
		(\subref{fig:metaicl-gewechselt}) is the result of applying WECHSEL
		\citep{minixhofer_wechsel_2022} to MetaICL.
		\textcolor[HTML]{357aa2}{GPT2-geWECHSELt+MetaICLA} combines
		\textcolor{Dandelion}{MetaICLA}, an adapter trained on the MetaICL dataset
		and objective, with a TGT-language GPT2 base obtained via WECHSEL.
		\textcolor[HTML]{40498e}{GPT2-geWECHSELt+MetaICLVA} does the same, except
		\textcolor{Dandelion}{MetaICLVA} is trained via targeted distillation with
		supervision provided by MetaICL. For more details, refer to section
		\ref{sec:method}.}
\end{figure*}

\section{Results and Discussion}

\begin{figure*}
	\includegraphics{baselines.pdf}
	\caption{Performance on a particular language dimension of our multi-task benchmark of our two
		baseline models, MetaICL and Sandwich. The dashed line separates whether a given task uses
		accuracy (left) or F1-score (right) as the performance metric.}
	\label{fig:results}
\end{figure*}

\begin{figure*}
	\includegraphics{results.pdf}
	\caption{Performance gap on our multi-task benchmark between each of the language-adapted models
		and the ``Sandwich'' baseline. Positive values indicate that the adapted models are
		outperforming the baseline, while negative values indicate the reverse. The dashed line
		separates whether a given task uses accuracy (left) or F1-score (right) as the performance
		metric.}
	\label{fig:results}
\end{figure*}

% Please add the following required packages to your document preamble:
% \usepackage{booktabs}
\begin{table}[ht]
	\centering
	\caption{Average performance across the datasets from our multi-task benchmark for the models
		considered in this work. We report average difference in performance for each proposed alternative
		to Sandwich. Negative values indicate underperformance compared to Sandwich.}
	\label{tab:results-summary}
	\begin{tabular}{@{}rccc@{}}
		\toprule
		\multicolumn{1}{c}{} & en    & fr     & de                             \\ \midrule
		MetaICL              & 0.327 & -      & -                              \\
		Sandwich             & -     & 0.317  & 0.322                          \\ \midrule
		\multicolumn{4}{c}{\textit{Difference in Performance w.r.t. Sandwich}} \\
		MetaICL-W            & -     & -0.020 & -0.026                         \\
		GPT2-W+MetaICLA      & -     & -0.041 & -0.042                         \\
		GPT2-W+MetaICLVA     & -     & -0.036 & -0.045                         \\ \bottomrule
	\end{tabular}
\end{table}

\section{Conclusion}

\bibliography{anthology,custom}

\section{Appendices}

Use \verb|\appendix| before any appendix section to switch the section numbering over to letters. See Appendix~\ref{sec:appendix} for an example.

\section{Example Appendix}
\label{sec:appendix}

This is an appendix.

\end{document}
